
\section{CERN and the LHC complex}
This is time for all good men to come to the aid of their party!

%\paragraph{Outline}
%The remainder of this article is organized as follows.
%Section~\ref{previous work} gives account of previous work.
%Our new and exciting results are described in Section~\ref{results}.
%Finally, Section~\ref{conclusions} gives the conclusions.

\subsection{CERN}
The European Center of Nuclear Research (in French: "\textbf{C}onseil \textbf{E}urop\a' een pour la \textbf{R}echerche \textbf{N}ucl\a' eaire", CERN) \cite{cern} is located in the Franco-Swiss border, in the outskirts of Geneva, Switzerland. It was founded in 1952 with the purpose of establishing a world-class fundamental physics research organization in Europe. There are currently 22 countries with full-member status, but scientists from all over the globe are working for the organization. CERN is comprised of many experiments and machines, most of which deal with the field of High Energy Physics (HEP). The largest experiments are ATLAS, CMS, LHCb and ALICE, which profit from the LHC which provides the base for these experiments to run.

\subsection{The Large Hadron Collider}
The Large Hadron Collider (LHC) is the world’s largest and most powerful particle accelerator. It first started up in 2008. The LHC consists of a 27-kilometre ring of superconducting magnets with a number of accelerating structures to boost the energy of the particles along the way.

Inside the accelerator, two high-energy particle beams travel at close to the speed of light before they are made to collide. The beams travel in opposite directions in separate beam pipes. They are guided around the accelerator ring by a strong magnetic field maintained by superconducting electromagnets. The electromagnets are built from coils of special electric cable that operates in a superconducting state, efficiently conducting electricity without resistance or loss of energy. This requires chilling the magnets to ‑271.3°C – a temperature colder than outer space. For this reason, much of the accelerator is connected to a distribution system of liquid helium, which cools the magnets, as well as to other supply services.

Thousands of magnets of different varieties and sizes are used to direct the beams around the accelerator. These include 1232 dipole magnets 15 metres in length which bend the beams, and 392 quadrupole magnets, each 5–7 metres long, which focus the beams. Just prior to collision, another type of magnet is used to "squeeze" the particles closer together to increase the chances of collisions. The particles are so tiny that the task of making them collide is akin to firing two needles 10 kilometres apart with such precision that they meet halfway.